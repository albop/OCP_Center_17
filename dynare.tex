%% LyX 2.2.2 created this file.  For more info, see http://www.lyx.org/.
%% Do not edit unless you really know what you are doing.
\documentclass[english]{beamer}
\usepackage[T1]{fontenc}
\usepackage[latin9]{inputenc}
\setcounter{secnumdepth}{3}
\setcounter{tocdepth}{3}
\usepackage{textcomp}
\usepackage{amsmath}

\makeatletter
%%%%%%%%%%%%%%%%%%%%%%%%%%%%%% Textclass specific LaTeX commands.
 % this default might be overridden by plain title style
 \newcommand\makebeamertitle{\frame{\maketitle}}%
 % (ERT) argument for the TOC
 \AtBeginDocument{%
   \let\origtableofcontents=\tableofcontents
   \def\tableofcontents{\@ifnextchar[{\origtableofcontents}{\gobbletableofcontents}}
   \def\gobbletableofcontents#1{\origtableofcontents}
 }
 \newenvironment{lyxcode}
   {\par\begin{list}{}{
     \setlength{\rightmargin}{\leftmargin}
     \setlength{\listparindent}{0pt}% needed for AMS classes
     \raggedright
     \setlength{\itemsep}{0pt}
     \setlength{\parsep}{0pt}
     \normalfont\ttfamily}%
    \def\{{\char`\{}
    \def\}{\char`\}}
    \def\textasciitilde{\char`\~}
    \item[]}
   {\end{list}}

\makeatother

\usepackage{babel}
\begin{document}

\title{Introduction to DSGE Models}
\makebeamertitle

\section{What are DSGE models?}

\section{What is Dynare?}
\begin{frame}{Dynare}
\begin{itemize}
\item Computes the solution of deterministic models
\item Computes first and second order approximation to solution of stochastic
models
\item Estimates parameters of DSGE models (with maximum likelihood or Bayesian
approach)
\item Computes optimal policy Performs global sensitivity analysis
\end{itemize}
We'll only focus on point 1 and 2. For the remaining points, you can
refer to the Dynare user guide.
\end{frame}
%
\begin{frame}{Deterministic models}
\begin{itemize}
\item These models are usually introduced to study the impact of a change
in regime, as in the introduction of a new tax, for instance.
\item Models assume full information, perfect foresight and no uncertainty
around shocks
\item Shocks can hit the economy today or at any time in the future, in
which case they would be expected with perfect foresight. They can
also last one or several periods.
\item Models introduce a positive shock today and zero shocks thereafter
(with certainty).
\item The solution does not require linearization, in fact, it doesn�t even
really need a steady state. Instead, it involves numerical simulation
to find the exact paths of endogenous variables that meet the model's
first order conditions and shock structure.
\item This solution method can therefore be useful when the economy is far
away from steady state (when linearization o�ers a poor approximation).
\end{itemize}
\end{frame}
%
\begin{frame}{Deterministic Models (see Juillard 1998)}
\begin{itemize}
\item We are in a perfect foresight framework, agents know about future
shocks.
\item The solution method is based on work of Lafargue, Boucekkine and Juillard.
\item The approximation imposes return to equilibrium in finite time (instead
of asymptotically) !
\begin{itemize}
\item Approximation of an infinite horizon model by a finite horizon one
\end{itemize}
\item Computes the trajectory of the variables numerically
\item The algorithm is a Newton-type method (Juillard 1996)
\end{itemize}
\end{frame}
%
\begin{frame}{Stochastic Models}
\begin{itemize}
\item In these models, shocks hit today (with a surprise), but thereafter
their expected value is zero.
\item Expected future shocks, or permanent changes in the exogenous variables
cannot be handled due to the use of Taylor approximations around a
steady state.
\item When these models are linearized to the 1st order, agents behave as
if future shocks where equal to zero (since their expectation is null),
which is the certainty equivalence property.
\item This is an often overlooked point in the literature which misleads
readers in supposing their models may be deterministic.
\end{itemize}
\end{frame}
%
\begin{frame}{Stochastic case: the general model}

\begin{align*}
E_{t}\left\{ f\left(y_{t+1},y_{t},y_{t-1},u_{t}\right)\right\}  & =0\\
E\left(u_{t}\right) & =0\\
E\left(u_{t}u_{t}^{\prime}\right) & =\Sigma_{u}\\
E\left(u_{t}u_{s}^{\prime}\right) & =0\,\forall t\neq s
\end{align*}

where $y_{t}$ is the vector of endogenous variables and $u_{t}$
the one of exogenous stochastic shocks.
\begin{itemize}
\item shocks $u_{t}$ are observed at the beginning of period $t$ 
\item not all variables are necessarily present with a lead and a lag
\item decisions affecting the current value of the variables $y_{t}$ are
function of the previous state of the system, $y_{t-1}$ and the shocks.
\end{itemize}
\end{frame}
%
\begin{frame}{A stochastic scale variable}

At period $t$, the only stochastic variable is $y_{t+1}$ and $u_{t+1}$.

We introduce the stochastic scale variable $\text{\ensuremath{\sigma}}$
and auxiliary random variable $\epsilon_{t}$ such that:

\begin{align*}
u_{t+1} & =\sigma\epsilon_{t+1}\\
E\left(\epsilon_{t}\right) & =0\\
E\left(\epsilon_{t}\epsilon_{t}^{\prime}\right) & =\Sigma_{\epsilon}\\
E\left(\epsilon_{t}\epsilon_{t}^{\prime}\right) & =0\,\forall t\neq s\\
\Sigma_{u} & =\sigma^{2}\Sigma_{\epsilon}
\end{align*}

\end{frame}
%
\begin{frame}{Solution Function}

\[
y_{t}=g\left(y_{t-1},u_{t},\sigma\right)
\]

If $\sigma=0$, the model is deterministic. One can prove the existence
of function $g$ and the conditions (see Jin and Judd, \textquotedbl{}Solving
Dynamic Stochastic Models\textquotedbl{}). Then:

\begin{align*}
y_{t+1} & =g\left(y_{t},u_{t+1},\sigma\right)\\
y_{t+1} & =g\left(g\left(y_{t-1},u_{t},\sigma\right),u_{t+1},\sigma\right)
\end{align*}

\[
F\left(y_{t-1},u_{t},u_{t+1},\sigma\right)=f\left(g\left(g\left(y_{t-1},u_{t},\sigma\right),u_{t+1},\sigma\right),g\left(y_{t-1},u_{t},\sigma\right),y_{t-1},u_{t}\right)
\]

\[
E_{t}\left\{ F\left(y_{t-1},u_{t},\sigma\epsilon_{t+1},\sigma\right)\right\} =0
\]

\end{frame}
%
\begin{frame}{Perturbation method}
\begin{itemize}
\item We need to obtain a Taylor expansion of the unknown solution function
in the neighborhood of a problem we can solve, i.e., the steady state.
\item The perturbations can be in the neighborhood of the ss OR increasing
$\sigma$ from zero.
\item The Taylor approximation of the solution is taken with respect to
$y_{t-1}$, $u_{t}$ and $\sigma$
\end{itemize}
\[
y_{t}=g\left(y_{t-1},u_{t},\sigma\right)
\]

\end{frame}
%
\begin{frame}{Deterministic Steady State}
\begin{itemize}
\item The Taylor approximation is taken with respect to the deterministic
steady-state, i.e. the one we obtain in absenc e of shocks. It satisfies:
\end{itemize}
\begin{align*}
f\left(\bar{y},\bar{y},\bar{y},0\right) & =0\\
\bar{y} & =g\left(\bar{y},0,0\right)
\end{align*}

Clearly, there can be multiple ss. but the approximation is taken
w.r.t. one only.
\end{frame}
%
\begin{frame}{Solution}
\begin{itemize}
\item Dynare takes the first and second order Taylor approximation of the
solution function
\item It creates a structural state space representation (matrix)
\item It computes the Schur decomposition.
\item It checks the Blanchard-Kahn conditions.
\item It selects the stable trajectory.
\item More details (see Juillard's material).
\end{itemize}
\end{frame}
%
\begin{frame}{Writing a \textquotedbl{}.mod\textquotedbl{} file}
\begin{itemize}
\item preamble: lists variables and parameters
\item model: spells out the model
\item steady state or initial value: gives indications to find the steady-state
of a model or the starting point for simulations or impulse response
functions based on the model's solution
\item shocks: defines the shocks to the system
\item computation: instructs Dynare to undertake specific operations (e.e.
forecasting estimating impulse response functions)
\end{itemize}
\end{frame}
%
\begin{frame}{Writing a \textquotedbl{}.mod\textquotedbl{} file (2)}
\begin{itemize}
\item Each instruction of the model must be terminated by a semi-colon (;),
although a single instructions can span two lines, if you need extra
space (just don't put a semi-colon)
\item You can comment a line with forward slashes (//) or an entire section
with /{*}, {*}/
\end{itemize}
\end{frame}
%
\begin{frame}{Example (from Collard 2001): stochastic case}
\begin{itemize}
\item Firms are producing a homogeneous final product that can be either
consumed or invested by means of capital and labor services. Firms
own their capital stock and hire labor supplied by the households.
Households own the firms. In each and every period three perfectly
competitive markets open 0 the markets for consumption goods, labor
services, and financial capital in the form of firm's shares.
\item Utility function:
\[
E_{t}\sum\beta\left(\ln c_{t}-\theta\frac{h_{t}^{1+\psi}}{1+\psi}\right)
\]
where $0<\beta<1$ is a constant discont factor, $c_{t}$ is consumption
in period $t$, $h_{t}$ is the fraction of total available time dvoted
to productive activity in period $t$, $\theta>0$ and $\psi>0$.
\end{itemize}
\end{frame}
%
\begin{frame}{An example (cont'd)}

There exists a central planner that maximizes the household's utility
functino subject to the following budget constraint:

\[
c_{t}+i_{t}=y_{t}
\]

Investment is used to form physical capital, which accumulates in
the standard form as:

\[
k_{t+1}=e^{b_{t}}i_{t}+(1-\delta)k_{t},\,0<\delta<1
\]

where $b_{t}$ is a shock affecting incorporated technological progress.

Output is produced by means of capital and labour services, relying
on a constant returns to scale technology, represented by the following
Cobb-Douglas production function ($0<\alpha<1$):

\begin{align*}
y_{t} & =A_{t}k_{t}^{\alpha}h_{t}^{1-\alpha}\\
A_{t} & =e^{a_{t}}
\end{align*}

\end{frame}
%
\begin{frame}{Dynare}
\begin{itemize}
\item We assume that the shocks to technology are distributed with zero
mean but display both persistence across time and correlation in the
current period. Let us consider the joint process $\left(a_{t},b_{t}\right)$
defined as:
\end{itemize}
\[
\left(\begin{array}{c}
a_{t}\\
b_{t}
\end{array}\right)=\left(\begin{array}{cc}
\rho & \tau\\
\tau & \rho
\end{array}\right)\left(\begin{array}{c}
a_{t-1}\\
b_{t-1}
\end{array}\right)+\left(\begin{array}{c}
\epsilon_{t}\\
v_{t}
\end{array}\right)
\]

\begin{itemize}
\item where:
\begin{align*}
E\left(\epsilon_{t}\right) & =E\left(v_{t}\right)=0\\
E\left(\epsilon_{t}\epsilon_{s}\right) & =0,\,t\neq s\\
E\left(v_{t}v_{s}\right) & =0,\,t\neq s\\
E\left(\epsilon_{t}v_{s}\right) & =0,\,t\neq s
\end{align*}
\end{itemize}
\end{frame}
%
\begin{frame}{Equilibrium equations}
\begin{itemize}
\item Arbitrage consuption/labour effort:
\item 
\[
c_{t}\theta h_{t}^{1+\psi}=\left(1-\alpha\right)y_{t}
\]
\item Euler:
\begin{align*}
\beta E_{t}\left[\left(\frac{e^{b_{t}}c_{t}}{e^{b_{t+1}}c_{t+1}}\right)\left(e^{b_{t+1}}\right)\alpha\frac{y_{t+1}}{k_{t+1}}+1-\delta\right] & =1
\end{align*}
\[
y_{t}=e^{a_{t}}k_{t}^{\alpha}h_{t}^{1-\alpha}
\]
\end{itemize}
\begin{lyxcode}
\[
k_{t+1}=e^{b_{t}}\left(y_{t}-c_{t}\right)+\left(1-\delta\right)k_{t}
\]

\begin{align*}
a_{t} & =\rho a_{t-1}+\tau b_{t-1}+\epsilon_{t}\\
b_{t} & =\tau a_{t-1}+\rho b_{t-1}+\nu_{t}
\end{align*}
\end{lyxcode}
\end{frame}
%
\begin{frame}{Preamble}
\begin{itemize}
\item endogenous variables: \textquotedbl{}var y, c, k, a, h, b;\textquotedbl{}
\item exogenous variables: \textquotedbl{}varexo e, u;\textquotedbl{}
\item the list of parameters: \textquotedbl{}parameters beta, rho, alpha,
delta, theta, psi, tau;\textquotedbl{}
\item Parameters values: 
\end{itemize}
\begin{lyxcode}
\textquotedbl{}alpha~=~0.36;\textquotedbl{}~(capital~elasticity~in~the~production~function)

\textquotedbl{}rho~=~0.95;\textquotedbl{}~(shock~persistence)

\textquotedbl{}tau~=~0.025;\textquotedbl{}(cross~persistence)

\textquotedbl{}beta~=~0.99;\textquotedbl{}(discount~factor)

\textquotedbl{}delta~=~0.025;\textquotedbl{}(capital~depreciation~rate)

\textquotedbl{}psi~=~0;\textquotedbl{}(labor~supply~elasticity)

\textquotedbl{}theta~=~2.95;\textquotedbl{}~(disutility~of~labor)

\textquotedbl{}phi~=~0.1;\textquotedbl{}(not~in~the~model~but~useful~to~express~the~variance~covariance~matrix)
\end{lyxcode}
\end{frame}
%
\begin{frame}{The model}
\begin{itemize}
\item Model declaration: It starts with the instruction model; and ends
with end;, in between all equilibrium conditions are written exactly
the way we write it by hand.
\item There need to be as many equations as you declared endogenous variables.
\item Equations are entered one after the other; no matrix representation
is necessary. Variable and parameter names used in the model block
must be the same as those declared in the preamble; variable and parameter
names are case sensitive.
\end{itemize}
\end{frame}
%
\begin{frame}{The model: timing conventions}
\begin{itemize}
\item If x is \textit{decided} in period t then we simply write x.
\item When the variable is decided in t-1, such as the capital stock in
our simple model, we write x(-1).
\item When a variable is decided in the next period, t + 1, such as consumption
in the Euler equation, we write x(+1).
\end{itemize}
\end{frame}
%
\begin{frame}{The model}

model;

c{*}theta{*}h\textasciicircum{}(1+psi)=(1-alpha){*}y;

k = beta{*}(((exp(b){*}c)/(exp(b(+1)){*}c(+1))) {*}(exp(b(+1)){*}alpha{*}y(+1)+(1-delta){*}k));

y = exp(a){*}(k(-1)\textasciicircum{}alpha){*}(h\textasciicircum{}(1-alpha));

k = exp(b){*}(y-c)+(1-delta){*}k(-1);

a = rho{*}a(-1)+tau{*}b(-1) + e;

b = tau{*}a(-1)+rho{*}b(-1) + u;

end;
\end{frame}
%
\begin{frame}{The model (for stochastic models)}
\begin{itemize}
\item initval;
\item y = 1.08068253095672; c = 0.80359242014163; h = 0.29175631001732;
k = 11.08360443260358; a = 0; b = 0; e = 0; u = 0; end; 
\item Adding steady just after your initval block will instruct Dynare to
consider your initial values as mere approximations and start simulations
or impulse response functions from the exact steady state. 
\item You need to enter a \textquotedbl{}good\textquotedbl{} steady state.
Dynare can help in finding your model's steady state by calling the
appropriate Matlab functions.
\end{itemize}
\end{frame}
%
\begin{frame}{The model: the check command}

after the initval or endval block (following the steady command if
you decide to add one): \textquotedbl{}check\textquotedbl{} command.
This computes and displays the eigenvalues of your system which are
used in the solution method. As mentioned earlier, a necessary condition
for the uniqueness of a stable equilibrium in the neighborhood of
the steady state is that there are as many eigenvalues larger than
one in modulus as there are forward looking variables in the system.
\end{frame}
%
\begin{frame}{Shocks (stochastic)}
\begin{itemize}
\item Can only be temporary and hit the system today (expectation of future
shocks must be zero).
\item We can make the efect of the shock propagate slowly throughout the
economy by introducing a latent shock variable $\epsilon_{t}$ and
$\nu_{t}$ that affects the model's true exogenous variable, $a_{t}$
and $b_{t}$ ,AR(1) 
\item In that case, though, we would declare $a_{t}$ and $b_{t}$ as endogenous
variables and $\epsilon_{t}$ and $\nu_{t}$ as exogenous variables
\end{itemize}
\begin{lyxcode}
shocks;~

var~e;~stderr~0.009;~

var~u;~stderr~0.009;

var~e,~u~=~phi{*}0.009{*}0.009;~

end;
\end{lyxcode}
\end{frame}
%
\begin{frame}{Computation (stochastic)}
\begin{itemize}
\item stoch\_simul is usually the appropriate command
\item it computes a Taylor approximation of the decision and transition
functions for the model (the equations listing current values of the
endogenous variables of the model as a function of the previous states
of the model and current shocks), impulse response functions and various
descriptive statistics (moments, variance decomposition, correlation
and autocorrelation coefficients)
\end{itemize}
\end{frame}
%
\begin{frame}{Computation (stochastic)}

Main options (see the Manual)
\begin{itemize}
\item irf=INTEGER: number of periods on which to compute the IRFS (default=40).
Setting IRF=0, suppresses the plotting of IRF's
\item order=1 or 2: order of Taylor approximation (default=2), unless you're
working with a linear model, in which case order is automatically
set to 1.
\end{itemize}
\end{frame}
%
\begin{frame}{Output (stochastic)}
\begin{itemize}
\item Model summary: a count of the various variable types in your model
(endogenous, jumpers, etc...)
\item If you use the check command: eigenvalues and a confirmation of the
Blanchard-Kahn conditions
\item Matrix of covariance of exogenous shocks (consistent with the input
on the shock)
\item Policy and transition functions
\item Moments of simulated variables (up to fourth moments)
\item Correlation of simulated variables (contemporaneous)
\item Autocorrelation of simulated variables: up to the f-th lag as specified
in the options of stoch simul.
\end{itemize}
\end{frame}
%
\begin{frame}{Deterministic model, shock}
\end{frame}

\framesubtitle{A temporary shock to TFP}
\begin{itemize}
\item $A_{t}\neq e^{a_{t}}$, $\epsilon_{t}=0$, $v_{t}=0$
\begin{itemize}
\item The economy is at the steady state (A=1)
\item There is an unexpected drop in TFP of 10\% at the beginning of period
1
\item Deterministic shocks are described in shocks block
\end{itemize}
\end{itemize}
%
\begin{frame}{Deterministic model, shock}
\end{frame}

\framesubtitle{A temporary shock to TFP}
\begin{lyxcode}
initval;

A=1;

y~=to~be~computed;

c~=~to~be~computed;

h~=~to~be~computed;

k~=~to~be~computed;

a~=~0;

b~=~0;

end;~

steady;

shocks;

var~A;

periods~1;

values~0.9;

end;
\end{lyxcode}
%
\begin{frame}{Deterministic model, shock}
\end{frame}

\framesubtitle{A temporary shock to TFP}
\begin{itemize}
\item the economy is at the steady-state
\item TFP jumps by 4\% in period 5 and grows by 1\% during the 4 following
periods
\end{itemize}
\begin{lyxcode}
shocks;

var~A;

periods~4,~5,~6,~7,~8;

values~1.04,~1.05,~1.06,~1.07,~1.08;

end;
\end{lyxcode}
%
\begin{frame}{Deterministic model, shock}
\end{frame}

\framesubtitle{A permanent shock}
\begin{itemize}
\item The economy is at the initial steady state (A=1)
\item In period 1, TFP jumps to 1.05, permanently
\end{itemize}
%
\begin{frame}{Deterministic model, shock}
\end{frame}

\framesubtitle{A permanent shock}
\begin{lyxcode}
model..remember~that~the~exo~variable~is~here~A.~

end;~

initval;

A=1;~...initial~steady~state..

end;

steady;

endval;

A=1.05;~..new~steady~state,~to~be~computed..~end;
\end{lyxcode}
%
\begin{frame}{Deterministic model, output}
\end{frame}

\framesubtitle{A permanent shock}
\begin{itemize}
\item Not very detailed
\item Steady state values, if command steady
\item Eigenvalues, if check
\item Some intermediate output: the errors at each iteration of the Newton
solver used to estimate the solution to your model
\item Sou need custom codeif you want richer statistics
\end{itemize}

\end{document}
\grid
